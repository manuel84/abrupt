
%%%%%%%%%%%%%%%%%%%%%%% paper_sem_web.tex %%%%%%%%%%%%%%%%%%%%%%%%%
%
% This is the LaTeX source for the instructions to authors using
% the LaTeX document class 'llncs.cls' for contributions to
% the Lecture Notes in Computer Sciences series.
% http://www.springer.com/lncs       Springer Heidelberg 2006/05/04
%
% It may be used as a template for your own input - copy it
% to a new file with a new name and use it as the basis
% for your article.
%
% NB: the document class 'llncs' has its own and detailed documentation, see
% ftp://ftp.springer.de/data/pubftp/pub/tex/latex/llncs/latex2e/llncsdoc.pdf
%
%%%%%%%%%%%%%%%%%%%%%%%%%%%%%%%%%%%%%%%%%%%%%%%%%%%%%%%%%%%%%%%%%%%


\documentclass[runningheads,a4paper]{llncs}
\usepackage[utf8x]{inputenc}
\usepackage{amssymb}

\setcounter{tocdepth}{3}
\usepackage{graphicx}
\usepackage{mathtools}
\usepackage{url}

\usepackage{listings}


\usepackage{xcolor}       
\definecolor{olivegreen}{rgb}{0.2,0.8,0.5}
\definecolor{grey}{rgb}{0.5,0.5,0.5}
\lstdefinelanguage{ttl}{
sensitive=true,
morecomment=[l][\color{grey}]{@},
morecomment=[l][\color{olivegreen}]{\#},
morestring=[b][\color{blue}]\",
}


\urldef{\mailsa}\path|manuel.b.dudda@student.hs-rm.de|    
\newcommand{\keywords}[1]{\par\addvspace\baselineskip
\noindent\keywordname\enspace\ignorespaces#1}

\begin{document}

\mainmatter  % start of an individual contribution

% first the title is needed
\title{Schließen auf Usability-Probleme mit OWL 2 RL}

% a short form should be given in case it is too long for the running head
\titlerunning{TODO: }

% the name(s) of the author(s) follow(s) next
%
% NB: Chinese authors should write their first names(s) in front of
% their surnames. This ensures that the names appear correctly in
% the running heads and the author index.
%
\author{Manuel Dudda}
%
\authorrunning{TODO:}
% (feature abused for this document to repeat the title also on left hand pages)

% the affiliations are given next; don't give your e-mail address
% unless you accept that it will be published
\institute{Hochschule RheinMain Informatik Master of Science \\
Fachbereich Design Informatik Medien \\
Campus Unter den Eichen 5
65195 Wiesbaden , Deutschland\\
\mailsa\\
\url{https://github.com/manuel84/abrupt}}

%
% NB: a more complex sample for affiliations and the mapping to the
% corresponding authors can be found in the file "llncs.dem"
% (search for the string "\mainmatter" where a contribution starts).
% "llncs.dem" accompanies the document class "llncs.cls".
%

\toctitle{Lecture Notes in Computer Science}
\tocauthor{Authors' Instructions}
\maketitle


\begin{abstract}
Die Usability von Internetauftritten wird größtenteils subjektiv wahrgenommen. 
Es existiert zwar eine Normierung durch die ISO für die Anforderungen an die Gebrauchstauglichkeit einer Software (EN ISO 9241-11), diese trifft jedoch nur sehr wage Aussagen und orientiert sich stark am respektiven Nutzungskontext. 
Trotzdem gibt es Merkmale einer Website, die allgemein als negativ für die Eigenschaften der Usability eingestuft werden können, so zum Beispiel schlechte Kontrastverhältnisse, ein geringer Lesbarkeitsindex und Diskrepanzen zwischen erwartetem und tatsächlichem Nutzerverhalten. 
Das "AbRUPt"-System der Hochschule RheinMain ist in der Lage viele dieser Merkmale einer Website automatisiert zu erfassen. 
Durch die Konvertierung, Modellierung und Entwicklung eines durchdachtes Regelsystem in der semantischen Ontologiesprache OWL 2 RL ensteht ein System, welches automatisiert Folgerungen auf Usability-Probleme ermöglicht.
\end{abstract}


\section{Einleitung}
\# TODO
\newpage
\section{Logik und Berechenbarkeit}

\# [WiP]

Mit der Aussagenlogik lassen sich beschränkt Formeln aufstellen. Sie entsprechen in der Programmierung dem Datentyp \textit{Boolean}, also Elementarussagen, die verknüpft und negiert erscheinen und deren Wahrheitsgehalt \textit{Wahr} oder \textit{falsch} ist.

\begin{align}
\begin{split}
A = & \textnormal{Die Checkbox A hat den Namen \textit{agb}}\\
B = & \textnormal{Die Checkbox ist ein Pflichtfeld}\\
A \to B = & \textnormal{Wenn eine Checkbox den Namen \textit{agb} hat,} \\
& \textnormal{dann ist sie ein Pflichtfeld}\\
A, A \to B = & \textnormal{Die Checkbox A hat den Namen \textit{agb}} \\
B = & \textnormal{Checkbox A ist ein Pflichtfeld}
\end{split}
\end{align}

Aus der oben genannten Regel (1), folgt, dass die Checkbox mit dem Namen \textit{agb} ein Pflichtfeld ist.
Eine gängige Darstellung solcher Regeln erfolgt durch Trennung der Prämisse und Konklusion mit einem Bruchstrich:

\begin{equation}
\frac{A, A \to B}{B}
\end{equation}

Über Wahrheitstafeln können jegliche (zusammengesetzte) Aussagen auf deren Wahrheitsgehalt überprüft werden.
Damit verdeutlicht sich einfach, dass das Schlussfolgern über die Aussagenlogik ein entscheidbares Problem darstellt.

\# TODO:

Eine Erweiterung der Aussagenlogik stellt die Prädikatenlogik dar.
In der Prädikatenlogik werden Elementaraussagen hinsichtlich ihrer Struktur untersucht.
Mit Quantoren \textit{forall, exist} und Prädikaten lassen sich Aussagen über die Existenz eines und Gültigkeit aller Individuen stellen.

Bsp. Prädikat

\begin{equation}
istPflichtfeld(x)
\end{equation}

Bsp. Quantor

\begin{equation}
\forall x. \neg Checkbox(x) \land (\neg istPflichtfeld(x) \lor istPflichtfeld(x))
\end{equation}

\begin{equation}
\exists x\exists y. Website(x) \land name(x, y)
\end{equation}

Somit lassen sich wesentlich mächtigere Aussagen treffen, die mit der Aussagenlogik allein nicht möglich sind.
Dies hat allerdings auch Asuwirkungen auf die Laufzeit des Schlussfolgerungs.
Das Entscheidungsverfahren für die Prädikatenlogik ist unentscheidbar.

Eine entscheidbare Untermenge der Prädikatenlogik erster Stufe bieten die Beschreibungslogiken.
Im Wesentlichen enthält eine Beschreibungslogik die Operatoren der Aussagenlogik (Negation, und, oder), ein- und zweistellige Prädikate sowie eine eingeschränkte Quantifizierung.
Sie umfassen Individuen, Eigenschaften und Klassen, wobei keine Klassen von Klassen möglich sind.
Normalerweie wird die Wissensbasis dabei in 2 Teile separiert:

TBox: Terminlogisches Wissen (Terminoligical Box)
Definiert das Vokabular in einem Anwendungsbereich (Terminologie, Grundvokabular), d.h. allgemeine Zusammenhange zwischen Objekten des Anwendungsbereichs.
Das Grundvokabular besteht aus Konzepten (einstellige Prädikate) und Rollen (zweistellige Prädikate)

ABox (Assertional Box)
Annahmen (assertions) über einzelne Objekte ausgedrückt mit
Hilfe des Grundvokabulars.

Mit dem RL-Profil aus OWL 2 (\cite{owl2rl}) existiert eine Sprache, die die gleiche Mächtigkeit wie die Beschreibungslogik (ALC...) besitzt und demnach für das Reasoning in entscheidbarer Komplexität bestens geeignet ist. 
OWL 2 ist zu dem Empfehlung des W3C und es existieren bereits einige Reasoner-Implementierungen. 
Die bekanntesten derzeit sind Fact++, Hermit, KAON2 und Pellet.

Axiome der Beschreibungslogik in OWL 2 RL
\begin{table}[h]
\begin{tabular}{|l|l|}
\hline
\textit{intersectionOf} & $ C_1 \sqcap ... \sqcap C_n$\\ \hline
\textit{unionOf} & $C_1 \sqcup ... \sqcup C_n$\\ \hline
\textit{complementOf} & $\neg C$\\ \hline
\textit{oneOf} & $ \{a_1\} \sqcup ... \sqcup \{a_n\}$\\ \hline
\textit{allValuesFrom} & $\forall P.C$\\ \hline
\textit{someValuesFrom} & $\exists P.C$\\ \hline
\textit{maxCardinality} & $ \leq n P$\\ \hline
\textit{subClassOf} & $ C_1 \sqsubseteq C_2 $\\ \hline
\textit{equivalentClass} & $ C_1 \equiv C_2 $\\ \hline
\textit{disjointWith} & $C_1 \sqsubseteq \neg C_2$\\ \hline
\textit{sameIndividualAs} & $ \{a_1\} \ sqsubseteq \{a_2\} $\\ \hline
\end{tabular}
\end{table}

Die exemplarisch genannten Regeln für die beschriebenen Logiken lassen sich somit in OWL 2 RL wie folgt ausdrücken.

\lstinputlisting[breaklines=true,language=ttl]{listings/propositional_logic_rule.ttl}

\lstinputlisting[breaklines=true,language=ttl]{listings/propositional_logic_infered.ttl}

\lstinputlisting[breaklines=true,language=ttl]{listings/description_logic_rule.ttl}

\lstinputlisting[breaklines=true,language=ttl]{listings/description_logic_infered.ttl}

\section{Implementierung}

\subsection{Anpassungen und Erweiterung des OWL-Converters}
\# TODO

In der Bachelor-Arbeit \cite{Brieger} wurde bereits untersucht inwieweit das Analyse-Tool \cite{url_abrupt} verwendet werden kann um eine objektive Beurteilung über die Benutzerfreundlichkeit einer Webseite geben zu können. 
Die Arbeit zeigt gute Ansätze und setzt auf die Konvertierung der erhobenen Daten in ein OWL-Dokument um anschließend mit einem durchdachten Regelsystem Schlussfolgerungen ziehen kann. 
Das Programm ist als Webseite gestaltet und kann lediglich die Webseiten-Daten umwandeln, die Integration der Webseiten-Besucherdaten ist nicht integriert. 
Auch die Integration in nachfolgende Projekte erweist sich durch den wenig modularen Aufbau als eher schwierig. 
Mit Einbezug der Ergebnisse aus \cite{Martin} konnte eine Konzept für eine sinnvolle Aufbereitung der Daten gezeigt werden. 
Es ist OWL-Converter entstanden, der erhobene \textit{AbRUPT}-Daten im XML-Format in ein OWL-Dokument transformiert.

\#TODO:warum eigentlich wdm-service?

Einige wichtige Aspekte wurden in dieser Arbeit geändert. 
Die Ontologie für den \textit{wdm-service} wurde von Grund auf intuitiver gestaltet und um die Konzepte der Webseiten-Besucherdaten erweitert (Fig. onto-Graph). 
Dabei wurde ebenso auf sinnvolle Konsistenzeigenschaften sowie Inferenz-Regeln zurückgegriffen, die auf Usabilty-Probleme schlussfolgern. (Abschnit Regeln)
Ein Problem ergab sich für die Eindeutigkeit von Webseiten-Elementen, die nicht kontextbasiert benannt wurden. 
Mit dem an \textit{REST} angelehnten URI-Aufbau, den Kontext in eingebetteter Struktur ergeben sich global einheitlich URIs. 
Ein Formular-Element kann somit auf mehreren Webseiten mit gleicher Id, Name und sonstigen Attributen existieren und wird nicht zu dem selben OWL-Element konvertiert. 
Beispielsweise liegt dann das Input-Textelement des Namens für ein Kontaktformular innerhalb von Formular (Form = \textit{kontaktformular}), Seitenstatus (State = \textit{state5}), Seiten-URL (Page = \textit{http://www.rikscha−mainz.de/Kontakt}) und der zugehörigen Website (Website = \textit{http://www.rikscha−mainz.de}).
Das sieht zunächst nicht sehr lesbar aus, ist aber aufgrund möglicher sehr großer Datenbestände unvermeidbar. 
\lstinputlisting[breaklines=true,language=ttl]{listings/unique_nested_uris.ttl}
 
- Errorstruktur
Problem
\subsection{Das abrupt-Gem (Converter-Lib + Binary)}
Für die Neugestaltung des Konvertierung-Tools fiel die Wahl der Technolgie auf das offizielle Paketsystem von Ruby \textit{RubyGems}. 
Zum einen ist die Unterstützung für RDF/OWL durch \cite{ruby-rdf} ausgezeichnet und zum Anderen erweisen sich noch weitere wesentliche Vorteile:
\begin{itemize}
\item{Benutzung durch Kommandozeilentool}
\item{Modularer Aufbau}
\item{Ausgezeichnete Unittest-Unterstützung}
\item{Wiederverwendbarkeit durch Einbindung als RubyGem Bibliothek oder des Kommandozeilentools}
\end{itemize}
\# TODO



- T-Box mit Protege (Vokabular + Regeln)

- A-Box: Instanzen durch Parameterübergabe der XML-Dateien an abrupt-binary Website-Data + CLient-Data

\newpage

\section{Schließen auf Probleme mit Inferenz-Regeln}
\subsection{Inkonsistenz-Regeln}
\subsection{Produktions-Regeln}
\subsection{Listen-Regeln}
\subsection{Datentyp-Regeln}

\section{Zusammenfassung \& Ausblick}
alter Converter:

- ~ 2000 Zeilen, kein Testcode

- in nur 4 Klassen

- Konfigurationen hartgecodet, schlecht wartbar
\\
neuer Converter:

- nur 800 Zeilen inkl. Crawlr (beta), Testcode

- Vokabular (T-Box) ausgelagert

- höchstens 130 Zeilen pro Klasse

- Rubygem: Kommandozeilentool, aber auch Code einbettbar in Ruby Web-Projekt


\nocite{url_dl_primer}

\bibliography{literature}
\bibliographystyle{alpha}

\end{document}
