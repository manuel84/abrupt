
%%%%%%%%%%%%%%%%%%%%%%% paper_sem_web.tex %%%%%%%%%%%%%%%%%%%%%%%%%
%
% This is the LaTeX source for the instructions to authors using
% the LaTeX document class 'llncs.cls' for contributions to
% the Lecture Notes in Computer Sciences series.
% http://www.springer.com/lncs       Springer Heidelberg 2006/05/04
%
% It may be used as a template for your own input - copy it
% to a new file with a new name and use it as the basis
% for your article.
%
% NB: the document class 'llncs' has its own and detailed documentation, see
% ftp://ftp.springer.de/data/pubftp/pub/tex/latex/llncs/latex2e/llncsdoc.pdf
%
%%%%%%%%%%%%%%%%%%%%%%%%%%%%%%%%%%%%%%%%%%%%%%%%%%%%%%%%%%%%%%%%%%%


\documentclass[runningheads,a4paper]{llncs}
\usepackage[utf8x]{inputenc}
\usepackage{amssymb}

\setcounter{tocdepth}{3}
\usepackage{graphicx}
\usepackage{mathtools}
\usepackage{url}

\usepackage{listings}


\usepackage{xcolor}       
\definecolor{olivegreen}{rgb}{0.2,0.8,0.5}
\definecolor{grey}{rgb}{0.5,0.5,0.5}
\lstdefinelanguage{ttl}{
sensitive=true,
morecomment=[l][\color{grey}]{@},
morecomment=[l][\color{olivegreen}]{\#},
morestring=[b][\color{blue}]\",
}


\urldef{\mailsa}\path|manuel.b.dudda@student.hs-rm.de|    
\newcommand{\keywords}[1]{\par\addvspace\baselineskip
\noindent\keywordname\enspace\ignorespaces#1}

\begin{document}

\mainmatter  % start of an individual contribution

% first the title is needed
\title{Schließen auf Usability-Probleme mit OWL 2 RL}

% a short form should be given in case it is too long for the running head
\titlerunning{TODO: }

% the name(s) of the author(s) follow(s) next
%
% NB: Chinese authors should write their first names(s) in front of
% their surnames. This ensures that the names appear correctly in
% the running heads and the author index.
%
\author{Manuel Dudda}
%
\authorrunning{TODO:}
% (feature abused for this document to repeat the title also on left hand pages)

% the affiliations are given next; don't give your e-mail address
% unless you accept that it will be published
\institute{Hochschule RheinMain Informatik Master of Science \\
Fachbereich Design Informatik Medien \\
Campus Unter den Eichen 5
65195 Wiesbaden , Deutschland\\
\mailsa\\
\url{https://github.com/manuel84/abrupt}}

%
% NB: a more complex sample for affiliations and the mapping to the
% corresponding authors can be found in the file "llncs.dem"
% (search for the string "\mainmatter" where a contribution starts).
% "llncs.dem" accompanies the document class "llncs.cls".
%

\toctitle{Lecture Notes in Computer Science}
\tocauthor{Authors' Instructions}
\maketitle


\begin{abstract}
Die Usability von Internetauftritten ist größtenteils eine subjektive Wahrnehmung. 
Es existiert zwar eine Normierung durch die ISO für die Anforderungen an die Gebrauchstauglichkeit einer Software (EN ISO 9241-11) doch diese ist sehr schwammig formuliert und orientiert sich stark am Nutzungskontext. 
Trotzdem gibt es Merkmale einer Website, die allgemein als Usability-Probleme eingestuft werden können, so zum Beispiel schlechte Kontrastverhältnisse, hoher Lesbarkeitsindex und Diskrepanzen zwischen erwarteten und tatsächlichen Besucherverhalten. 
Das "AbRUPt"-System der Hochschule RheinMain ist in der Lage viele dieser Merkmale einer Website automatisiert zu erfassen. 
Durch die Konvertierung, Modellierung und ein durchdachtes Regelsystem in der semantischen Ontologiesprache OWL 2 RL ensteht ein System, welches Folgerungen auf Usability-Probleme ermöglicht.
\end{abstract}


\section{Einleitung}
\# TODO
\newpage
\section{Logik und Berechenbarkeit}

\# [WiP]

Mit der Aussagenlogik lassen sich beschränkt Formeln aufstellen. Sie entsprechen in der Programmierung dem Datentyp "Boolean", also Elementarussagen, die verknüpft und negiert erscheinen können und deren Wahrheitsgehalt Wahr oder falsch.

\begin{align}
\begin{split}
A = & \\
& \textnormal{Das Kontrollkästchen hat den Namen \textit{AGB} und ist kein Pflichtfeld}
\end{split}\\
\begin{split}
A \to B = & \\
& \textnormal{Wenn das Kontrollkästschen den Namen \textit{AGB} hat und kein Pflichtfeld ist,} \\
& \textnormal{gibt es ein Usability-Problem.}
\end{split}\\
\begin{split}
A, A \to B = & \\
& \textnormal{Das Kontrollkätschen hat den Namen \textit{AGB} und ist kein Pflichtfeld,} \\
& \textnormal{es gibt ein Usability-Problem}
\end{split}
\end{align}

\lstinputlisting[breaklines=true,language=ttl]{listings/propositional_logic.ttl} 


%\begin{align}
%\begin{split}
%\url{wdm:Website/http://www.rikscha-mainz.de/Page/http://www.rikscha-mainz.de/Kontakt/State/state5/Form/32a0380204d524f2f1c9d3a3222a0a2a/Checkbox/agb} 
%a wdm:Checkbox; \\
%wdm:required true^^boolean 
%\end{split}
%\end{align}

Über Wahrheitstafeln können jegliche (zusammengesetzte) Aussagen auf deren Wahrheitsgehalt überprüft werden. Damit verdeutlicht sich einfach, dass das Schlussfolgern über die Aussagenlogik ein entscheidbares Problem darstellt.

\# TODO:

- Prädikatenlogik mit Beispiel (Unentscheidbarkeit)

- Beschreibungslogik (Einschränkungen bez. Prädikatenlogik, Entscheidbarkeit, OWL 2 RL)

\newpage

\section{Website-Usability und der \textit{AbRUPt}-Web-Service}

\# TODO maybe section 2, nach Einleitung
\section{Implementierung}

\subsection{Anpassungen und Erweiterung der Arbeit von Brieger}
\# TODO

- Entfernen der Klassenhierarchie, da nicht intuitiv.

- Erweitern der Object- und Dataproperties um Restriktionen und Eigenschaften (maxCardinality, functional, transitivity...)

- Anpassen der Ontology IRI aus unique Gründen (Vermeiden doppeltes )

- IRI-Aufbau: {wdm}/{type}/{page}{id}

- Inferbeispiel

- Errorstruktur
Problem
\subsection{Das abrupt-Gem (Converter-Lib + Binary)}
\# TODO

- T-Box mit Protege (Vokabular + Regeln)

- A-Box: Instanzen durch Parameterübergabe der XML-Dateien an abrupt-binary Website-Data + CLient-Data

\newpage

\section{Schließen auf Probleme mit Inferenz-Regeln}
\subsection{Triple-Pattern-Rules}
\subsection{Inkonsistenz-Regeln}
\subsection{Listen-Regeln}
\subsection{Datentyp-Regeln}

\section{Zusammenfassung \& Ausblick}
alter Converter:

- ~ 2000 Zeilen, kein Testcode

- in nur 4 Klassen

- Konfigurationen hartgecodet, schlecht wartbar
\\
neuer Converter:

- nur 800 Zeilen inkl. Crawlr (beta), Testcode

- Vokabular (T-Box) ausgelagert

- höchstens 130 Zeilen pro Klasse

- Rubygem: Kommandozeilentool, aber auch Code einbettbar in Ruby Web-Projekt





\begin{thebibliography}{4}

\bibitem{url_abrupt}AbRUPt,
Letzter Zugriff: 12. Dezember 2014\\
\url{http://wba.cs.hs-rm.de/AbRUPt/service/}
\bibitem{url_dl_primer}A Description Logic Primer\\
\url{http://arxiv.org/pdf/1201.4089v3.pdf}

\end{thebibliography}

\end{document}
